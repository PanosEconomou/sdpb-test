\documentclass{homework}
\title{SDPB Training}
\author{Panos Oikonomou}

\begin{document}
\maketitle
\problem[1 | Intro]

Studying $(\beta\partial)^n$ we can obtain that
\begin{equation*}
	(\beta\partial)^n = \sum_{k=1}^{n} S^n_k \beta^k \partial^k,
\end{equation*}
where $S^{n}_{k}$ is the number of partitions of a set of $n$ elements into $k$ disjoint subsets.

To show this we do some combinatorics. Effectively we want to expand $\beta \partial (\beta \partial(\beta \cdots \partial (\beta \partial f)))$ for some function $f(\beta)$ in terms of $f^{(k)}(\beta)$. Using product rule we know that the coefficient multiplying $f^{(k)}$ is going to be $\beta \partial (\beta \partial(\beta \cdots \partial (\beta)))$ with $k$ of the partials missing. Now we have $n-1$ positions to place $n-k$ partials in the string above, and if one is placed at position $1\leq i\leq n-1$ then it can act only to the $n-i$ remaining $\beta$'s. 

We can depict this pictorially using two sets of dots one set for the available position to place the $\partial$ and another for each $\beta$ that the $\partial$ can act on. Fig.~\ref{fig:dots} shows a representation of one of the terms in the product rule expansion when we place two derivatives on the first two slots.
\begin{figure}[h]
     \centering
     \includegraphics[width=0.5\linewidth]{dots}
	 \caption{\textit{Connecting the dots. The bottom dots represent places in the $\beta \partial (\beta \partial(\beta \cdots \partial (\beta)))$ string where $\partial$ can be placed, while the top dots represent the $\beta$ that are there. A connected line means that a corresponding $\partial$ acts on that particular $\beta$. The particular configuration corresponds to the term in the expansion with two derivates acting like so $\beta (\partial \beta) \beta (\partial \beta) \cdots \beta$.}}
     \label{fig:dots}
\end{figure}

Alternatively we can consider this as a set of $n$ elements with $n-k$ possible connections. The connectionspartition that set of points into $n-(n-k + 1) + 1 = k$ subsets, and each subset partition corresponds to exactly one term in the product rule expansion of the coefficient in front of $f^{(k)}$, therefore each term will be multiplied by the number of partitions of a set of $n$ elements into $k$ subsets, $S^n_k$.

Now the hard part is done. Using the fact that $\partial^{k} e^{-\beta H} = (-H)^{k}e^{-\beta H}$ we can conclude that 
\begin{align*}
	\alpha(Z(\beta)) = \text{Tr\,}\sum_{n=0}^{N} \alpha_n e^{-\beta H}\sum_{k=1}^{n} (-\beta H)^{k} S^{n}_{k} = \text{Tr\,} e^{-\beta H}\sum_{n=0}^{\lfloor\frac{N}{2}\rfloor} \alpha_{2n} T^n(-\beta H) + e^{-\beta H}\sum_{n=0}^{\lfloor\frac{N}{2}\rfloor} \alpha_{2n+1} T^n(-\beta H).
\end{align*}
If we call $\tilde \beta = \frac{4\pi^2}{\beta}$ we know that $Z(\beta) = Z(\tilde \beta)$ and that $\beta \partial = -\tilde \beta \tilde \partial$ plugging these two equations we have that 
\begin{align*}
	\alpha(Z(\beta)) = \alpha(Z(\tilde \beta)) = \text{Tr\,} e^{-\tilde\beta H}\sum_{n=0}^{\lfloor\frac{N}{2}\rfloor} (-1)^{2n} \alpha_{2n} T^n(-\tilde\beta H) + e^{-\tilde\beta H}\sum_{n=0}^{\lfloor\frac{N}{2}\rfloor} (-1)^{2n+1} \alpha_{2n+1} T^n(-\tilde\beta H).
\end{align*}
Subtracting the two expressions for $\beta = 2\pi = \tilde \beta$ we conclude that
\begin{align*}
	\text{Tr\,}\alpha(H) e^{-2\pi H} = \text{Tr\,}\sum_{n=0}^{\lfloor\frac{N}{2}\rfloor} (-1)^{2n+1} \alpha_{2n+1} T^n(-2\pi H)e^{-2\pi H} = 0.
\end{align*}
which is different form what the problem set we should get, but the rest of the considerations must be fine. Still if $\alpha(E) > 0$ for all allowed $E$ then the theory is excluded.
\problem*[2 | Implementation]
We know that a spectrum for a CFT is not valid if we can find a functional $\alpha$ such that $\text{Tr\,}\alpha(H) \geq 0$. So we can solve the optimization problem for $\alpha$ constraining all energies above the ground to give positive contributions, and solve for what the maximum $\alpha(E_0)$ is. If it positive then we have found a functional that contradicts the assertion above and then we can reject that spectrum. 

However, one can notice quickly that a possible solution to this problem, as stated in the original document, is to take $\alpha_n = 0$ for all $n$. Therefore, sdpb should always be able to find a polynomial that doesn't allow the theory to be rulled out. 
\end{document}
